\documentclass[]{report}
\usepackage[paperwidth=8.5in, paperheight=11in]{geometry}
\usepackage{enumerate}
\usepackage{color}
\usepackage{soul}
\renewcommand{\contentsname}{ }
\usepackage[hidelinks]{hyperref}
\usepackage{graphicx}

% Title Page
\title{
	\includegraphics[width=1.00\columnwidth]{header}\\
	\vspace{3.0 in}
	Constitution\\of the\\Dalhousie Computer Science Graduate Society\\(CSGS)
}
\author{ }
\date{Last Modified:\\\st{10 March 2014}\\\color{red}{13 March 2015}}

\begin{document}
\maketitle

\clearpage
\begin{center}
	\section*{Preamble}
	\vspace{12px}
\end{center}
\label{preamble}
	This document is the official constitution of the Dalhousie University Computer Science Graduate Society (CSGS). This document renders all previous Constitutions for the Society null and void.


\clearpage
\color{red}{
	\tableofcontents
}
\color{black}

\clearpage
\begin{center}
	\section*{Article 1:\\NAME}
	\addcontentsline{toc}{chapter}{Article 1: NAME}
	\vspace{12px}
\end{center}
\label{name}
	The society shall be named ``Dalhousie University Computer Science Graduate Society", abbreviated as CSGS and hereinafter referred to as the ``Society".


\clearpage
\begin{center}
	\section*{Article 2:\\PURPOSE}
	\addcontentsline{toc}{chapter}{Article 2: PURPOSE}
	\vspace{12px}
\end{center}
\label{purpose}

	The purpose of the Society is to:\\
	
	\renewcommand{\theenumi}{\Alph{enumi}}
	\begin{enumerate}
	
		\item Promote activities for the advancement of the interests of its members and other interested parties;
		
		\item Co-ordinate and promote activities of the member(s) of the Society subject to the rules and regulations of Dalhousie University;
		
		\item Present a unified voice of the membership of the Society in discussions both inside and outside Dalhousie University, when such discussions may affect the membership of the Society either directly or indirectly; and,
		
		\item Organize the money and properties granted to, or otherwise acquired by, the Society to satisfy its purpose;

	\end{enumerate}


\clearpage
\begin{center}
	\section*{Article 3:\\MEMBERSHIP}
	\addcontentsline{toc}{chapter}{Article 3: MEMBERSHIP}
	\vspace{12px}
\end{center}
\label{membership}

	\renewcommand{\theenumi}{\Alph{enumi}}
	\begin{enumerate}
		\item Any graduate student enrolled within the Faculty of Computer Science, Dalhousie University, who pays the prescribed Society Fee shall be eligible to be a member of the Society, during the term in which the member's fee was collected.
		
		\item Members shall pay the Society Fee to maintain their status.
		
		\item Only members of the Society shall vote and be eligible to hold Council positions.
		
		\item The following members shall be eligible to become honorary members of the Society:
	
		\begin{enumerate}[i.]
		
			\item Any person taking a graduate course offered by the Faculty of Computer Science for credit;
			
			\item The Dean of the Faculty of Computer Science, Dalhousie University;
			
			\item Faculty members of the Faculty of Computer Science, Dalhousie University;
			
			\item Graduate student alumni of the Faculty of Computer Science, Dalhousie University;
			
			\item Any other person(s) elected by the Council of the Society;
			
		\end{enumerate}
	
		\item Honorary members shall not have voting privileges and shall not be eligible to hold Executive positions in the Council of the Society.
		
		\item The Society shall not place any limits on membership based on, but not limited to, age, gender, race, religion, language, sexual orientation, or disability. The Society is committed to providing an inclusive and welcoming environment for all members.
		
	\end{enumerate}
	

\clearpage
\begin{center}
	\section*{Article 4:\\ORGANIZATION}
	\addcontentsline{toc}{chapter}{Article 4: ORGANIZATION}
	\vspace{12px}
\end{center}
\label{organization}
	\renewcommand{\theenumi}{\Alph{enumi}}
	\begin{enumerate}
	
		\item There shall be a council of students (hereinafter referred to as the ``Council"), which shall consist of the administrative positions (hereinafter referred to as the ``Executive") and elected or appointed representatives.
		
		\item The Executive shall consist of FIVE elected officers. A member can be elected to hold only ONE Executive position at a time. In order of ascension, these officers shall be:
		
		\begin{enumerate}[i.]
		
			\item President
			\item Vice-President Internal
			\item Vice-President External
			\item Treasurer
			\item Secretary
		
		\end{enumerate}
		
		\item The Council shall consist of:
		
		\begin{enumerate}[i.]
				
			\item The Executive
			\item PhD Representative
			\item MCS Representative
			\item MACS Representative
			\item MHI Representative
			\item MEC Representative
			\item Member-At-Large
			\item Chair (Non-Voting)
			\item Faculty Representative (Non-Voting)
			\item Transition Officer (Non-Voting)
		
		\end{enumerate}
		
		\item The representatives of the PhD, MCS, MACS, MHI, MEC programs shall hereinafter be referred to as the ``Program Representatives".
		
		\item Members are eligible to contest for any of the Council positions only if they are registered students at Dalhousie University for the entire term in office.
		
		\item A member will NOT be eligible to contest for any of the Executive positions if they will be unavailable on campus during business hours, owing to Co-op, internship, or similar arrangement.
			
	\end{enumerate}


\clearpage
\begin{center}
	\section*{Article 5:\\DUTIES OF THE COUNCIL}
	\addcontentsline{toc}{chapter}{Article 5: DUTIES OF THE COUNCIL}
	\vspace{12px}
\end{center}
\label{duties}
	\renewcommand{\theenumi}{\Alph{enumi}}
	\begin{enumerate}
	
		\item Subject to the By-Laws, the Council may appoint and delegate to committees such powers and duties as maybe deemed necessary or desirable. The Council may also make rules and regulations relating to the performance of its functions and the exercise of its powers.
		
		\item Subject to Article 16, the Council may introduce, amend, or repeal any By–Law in accordance with this constitution.
		
		\item The Council may delegate responsibilities and duties to Council members including at least the applicable duties in Article 6. This delegation shall not limit the oversight of Council as expressed in Article 5(D).
		
		\item The Council as a whole may hold a vote to overrule any executive or committee decision.
		
		\item The Council must ensure that the Society begins the ratification process by May 1st of each calendar year, as recommended by the DSU, and ensure the Society is ratified no later than August 1st of that year.
	
	\end{enumerate}


\clearpage
\begin{center}
	\section*{Article 6:\\DUTIES OF COUNCIL MEMBERS}
	\addcontentsline{toc}{chapter}{Article 6: DUTIES OF COUNCIL MEMBERS}
	\vspace{12px}
\end{center}
\label{dutiesCouncilMembers}

	\renewcommand{\theenumi}{\Alph{enumi}}
	\begin{enumerate}
		\item The duties of each Executive Officer shall be as follows:\\
	
		\begin{enumerate}[i.]
			\item President:
			
			\begin{enumerate}[(a)]
				\item Represent the Society as the head of the Executive.
				\item Ensure the representation of the Society to all external organizations.
				\item Be responsible for facilitating the creation of an annual vision for the Society, in consultation with the Executive.
				\item Fulfil the position of the Chair:
					\begin{enumerate}[(1)]
						\item In the absence of the Chair during general Council meetings;
						\item During Executive meetings; and,
						\item In the absence of the Chair during annual general meetings.
					\end{enumerate}
				\item Be responsible for overseeing all operations of the Society.
				\item Sit as an ex-officio member of all committees of the Society.
				\item Maintain a file of the minutes to all the committees of which he or she is a part, and to pass this file, a summary of the contents of this file to the appointed transition officer at the end of the current year.
				\item In the absence of a transition officer, the President shall act as the transition officer, and perform the functions as per Section 6.B.v.
				\item Delegate his or her duties where appropriate or necessary.
				\item Attend Dalhousie Student Union Society Training Day at the date specified by the Vice-President (Internal) of the DSU.
				\item Act as a signing officer of the Society.
			\end{enumerate}
		
			\item Vice-President Internal:
			\begin{enumerate}[(a)]
				\item Act as a liaison between the Faculty of Computer Science and Society membership.
				\item Sit on the Dean's Executive, the Faculty Council, the Graduate Committee, and other Faculty of Computer Science committees as the Graduate Student Representative for Computer Science.
				\item Deal with all internal issues faced by the Society including courses, faculty and curriculum.
				\item Act as chair for general Council meetings in the absence of the Chair and the President.
				\item May delegate his or her responsibilities on Faculty committees to the respective Program Representative, should a matter for discussion at the committee be pertinent to their respective constituents;
				\item Delegate his or her duties where appropriate or necessary.
				\item Attend Dalhousie Student Union Society Training Day at the date specified by the Vice-President (Internal) of the DSU.
				\item Act as a signing officer of the Society.
			\end{enumerate}
		
			\item Vice-President External:
			\begin{enumerate}[(a)]
				\item Act as a liaison with the Dalhousie University Computer Science Society.
				\item Act as a liaison with all organizations external to the Faculty of Computer Science.
				\item Deal with all external Society issues including non-campus activities, and community service.
				\item Attend Dalhousie Student Union meetings and represent the Society on appropriate Dalhousie Student Union committees.
				\item Attend meetings of the Dalhousie Association of Graduate Students and represent the Society on appropriate Dalhousie Association of Graduate Students committees.
				\item Facilitate ratification for the Society and adhere to Article 5(E).
				\item Act as chair for general Council meetings in the absence of the Chair, President and Vice-President Internal.
				\item Delegate his or her duties where appropriate or necessary.
				\item Attend Dalhousie Student Union Society Training Day at the date specified by the Vice-President (Internal) of the DSU.
				\item Act as a signing officer of the Society, in case the Vice-President Internal is not able to exercise his or her duties as a signing officer.
			\end{enumerate}
		
			\item Treasurer:
			\begin{enumerate}[(a)]
				\item Keep records of all spending within the Society in accordance with DSU regulations.
					\begin{enumerate}[(1)]
						\item Maintain accurate records of all receipts and expenditures of the Society.
						\item Maintain monthly bank reconciliation, two weeks after receipts are dated and signed.
						\item Provide a statement of account every month.
						\item Prepare a report every semester showing the financial position of Council.
					\end{enumerate}
				\item Prepare a file at the end of the year summarizing the past year, to be given to the transition officer. This file should consist of:
					\begin{enumerate}[(1)]
						\item A report of all revenues and expenditures for the previous year;
						\item A report of the financial position of the Society;
						\item A proposed budget for the next year;
						\item An outline of the accounting procedures used in the Society; and,
						\item Guidance for the incoming treasurer on how best to handle the accounts of the Society.
					\end{enumerate}
				\item Ensure that the books are audited by the Dalhousie Student Union Vice-President (Finance) and the Dalhousie Association of Graduate Students.
				\item Report revenue for all events organized by Council in coordination with the organizer of the event.
				\item Chair a meeting of the Executive to estimate the budget for each newly elected Council in accordance with the guidelines set forth in Article 12 of this Constitution.
				\item Ensure that the Council shall meet at least once per semester to make available to its members the following:
					\begin{enumerate}[(1)]
						\item A detailed summary of the financial statements for the preceding semester, audited at least internally;
						\item A detailed budget of the Council and the committees under its jurisdiction for the current year. This budget must be presented within first month of each newly elected Council;
					\end{enumerate}
				\item Attend Dalhousie Student Union Society Training Day at the date specified by the Vice-President (Internal) of the DSU.
				\item Act as a signing officer of the Society.
			\end{enumerate}
		
			\item Secretary:
			\begin{enumerate}[(a)]
				\item Record the minutes at Council and Executive meetings.
				\item Distribute the minutes to members before or at the next meeting.
				\item Give advance notice of all motions to be voted on at any meeting not less than FORTY-EIGHT (48) hours before the meeting, and this notice must be given to all members of the Society.
				\item Prepare all correspondence on behalf of the Society on the recommendations of the Executive or Council.
				\item Advertise vacant positions to the Society whenever such positions are open.
				\item Co-ordinate all publicity on behalf of Council.
				\item Maintain the Society's web presence.
			\end{enumerate}
		\end{enumerate}
	
		\clearpage
		\item The duty of other Council members are as follows:\\

		\begin{enumerate}[i.]
			\item Program Representatives:
				\begin{enumerate}[(a)]
					\item Represent their respective student bodies at Council meetings;
					\item Appoint acting representatives to the meetings in his or her absence.
					\item Attend Faculty committee meetings on behalf of their constituents and the Society, when the matter under discussion at the committee be pertinent to their respective constituents, and when such responsibility has been delegated by the Vice-President Internal, and inform the Council and their constituents of the proceedings after each committee meeting.
					\item Keep their respective constituents informed of the issues discussed during the Council meetings.
					\item Make known to the Council any complaints, comments, suggestions, or ideas that their constituents may have.
					\item Encourage constituent participation in social events.
				\end{enumerate}

			\item Member-At-Large:
				\begin{enumerate}[(a)]
					\item Represent the Graduate Student body at the Faculty of Computer Science as a whole; and,
					\item Act on the direction of the Council.
				\end{enumerate}
				
			\item Chair (non-voting):
				\begin{enumerate}[(a)]
					\item Selected by the Council.
					\item Giving notice of and preparing an agenda for all general Council meetings and AGMs (referred together as “meetings” in this section).
					\item Book a room to hold the general Council meetings and AGMs.
					\item Possess thorough knowledge of the Constitution and Robert’s Rules.
					\item Chair meetings in accordance to the Constitution and Robert’s Rules.
					\item Ensure meetings are conducted in a timely and efficient manner.
					\item Ensure fair representation of all members during meetings.
				\end{enumerate}

			\item Faculty Representative (non-voting):
				\begin{enumerate}[(a)]
					\item Selected by the Faculty Council.
					\item Attend Council meetings, as required, to provide guidance and input to Council.
%					\item Help transition between consecutive Councils by providing guidance to the incoming Council as to what has been done by the outgoing Council.
				\end{enumerate}

			\clearpage
			\item Transition Officer:
				\begin{enumerate}[(a)]
					\item Selected by the Council.
					\item Transition between Councils and provide guidance for the new Council.
					\item Collect reports outlined in Article 6(D), End of Year Reports and possessions of the Society from the previous Council.
					\item Deliver all collected reports and possessions of the Society to the new Council.
				\end{enumerate}
		\end{enumerate}
		
		\item Duties for other positions may be outlined in the By-Laws.
		
		\item The duties for all Council Members shall be as follows:
			\begin{enumerate}[i.]
				\item Attend Council meetings.
				\item Submit an End of the Year Report, as outlined in Article 12 summarizing their activities and progress on Council along with any Council assets he or she may hold, to the transition officer prior to the elections.
			\end{enumerate}

	\end{enumerate}


\clearpage
\begin{center}
	\section*{Article 7:\\MEETINGS}
	\addcontentsline{toc}{chapter}{Article 7: MEETINGS}
	\vspace{12px}
\end{center}
\label{meetings}
	\renewcommand{\theenumi}{\Alph{enumi}}
	\begin{enumerate}
		
		\item Meetings follow Robert's Rules of Order.
		\item There shall be three types of meetings: general meetings, executive meetings, and annual general meetings (AGM’s).
		\item The regulations regarding each type of meeting are as follows:
			\begin{enumerate}[i.]
				\item General council meetings:
					\begin{enumerate}[(a)]
						\item \st{Quorum for general meetings shall be $ (2/3) ^{rd}$ of the executive and $ (2/3) ^{rd}$ of the (non-executive) council members.}\color{red}\\
						Quorum for general council meetings shall be $ (50\% + 1) $ of the council members.
						\item In case a council member is not able to attend the general council meeting due to unavoidable circumstances, the member shall convey regrets to the Chair of the council, and may nominate a council member as `proxy' to cast votes on his/her behalf. The functions of the proxy shall be as follows:
							\begin{enumerate}[(1)]
								\item The proxy shall be included as the absentee member's presence in the quorum count; 
								\item In voting on any motions during the meeting (as highlighted by Article 7(C)(1)(e)), the council member nominated as the proxy shall:\\
									  I. \space Base the proxy vote on the directions of the absentee council member; or,\\
									  II. Use his/her discretion and best judgement to vote on behalf of the absentee council member.
							\end{enumerate}
							 \color{black}
						\item Any member of the general membership shall be welcome to attend the general meetings.
						\item Voting on motions put forth at general meetings must go before the general membership.
						\item Voting shall be conducted in one of three ways:
							\begin{enumerate}[(1)]
								\item Show of hands;
								\item Roll call, if requested by any member of the Council; or,
								\item Secret ballot, if requested by any member of the Council;
							\end{enumerate}
						\item A majority vote is needed to pass a motion (a majority signifying 50\%+1 of all voting members in attendance).
						\item The Secretary of the Society shall be responsible for giving advance notice of all motions to be voted on at any meeting not less than FORTY-EIGHT  (48) hours before the meeting, and this notice must be given to all members of the society.
					\end{enumerate}
				\item Executive meetings:
					\begin{enumerate}[(a)]
						\item \st{Quorum for executive meetings shall be $ (2/3) ^{rd}$ of the executive.}
					\end{enumerate}
					\begin{enumerate}[(a)]
						\item Executive meetings shall be used at the discretion of the executive to ensure the smooth operation of the Society.
						\item All members of the Society are permitted to attend Executive meetings, even though notice of the meeting need not be given.
						\item Any motions must be voted on by the general membership, and as such motions cannot be passed during Executive meetings.
					\end{enumerate}
				\item Annual general meetings (AGM's)
					\begin{enumerate}[(a)]
						\item Quorum for an AGM shall be all Executive members along with 10\% of the general membership.
							\color{red}
						\item In case a council member is not able to attend the AGM due to extreme circumstances, including but not limited to illness, thesis presentation, conference travel, etc., the member shall convey regrets to the President, and may nominate a council member as `proxy' to cast votes on his/her behalf. The functions of the proxy shall be as follows:
							\begin{enumerate}[(1)]
								\item The proxy shall be included as the absentee member's presence in the quorum count; 
								\item In voting on any motions during the meeting, the council member nominated as the proxy shall:\\
									  I. \space Base the proxy vote on the directions of the absentee council member; or,\\
									  II. Use his/her discretion and best judgement to vote on behalf of the absentee council member.
							\end{enumerate}
							 \color{black}
						\item An AGM must be called at least once per academic year by the President.
						\item There will be at least one AGM during the months of February or March, during which elections may take place (see Article 8). 
						\item Any additional AGM's may be called at the request of all Executive, or at the request of a member of the general membership with a petition for an AGM signed by 50\% of the general members.
						\item Constitutional amendments can only be made at an AGM (see Article 11).
						\item Voting on motions must go before the general membership.
						\item Voting will be conducted via a show of hands, though any member may request that the motion be voted on by secret ballot. If any one member requests this for any motion, voting must be done by secret ballot.
						\item A majority vote \color{red}{of those present and voting (including proxies)} \color{black} is needed to pass a motion.
						\item Notice of an AGM must be given to all members not less than one week before the AGM, and the Secretary is responsible for delivering this notice.
					\end{enumerate}
			\end{enumerate}
			
	\end{enumerate}


\clearpage
\begin{center}
	\section*{Article 8:\\ELECTIONS}
	\addcontentsline{toc}{chapter}{Article 8: ELECTIONS}
	\vspace{12px}
\end{center}
\label{elections}
	\renewcommand{\theenumi}{\Alph{enumi}}
	\begin{enumerate}
	
		\item Elections shall be held no later than March 31 of each year.
		\item The President shall be responsible for overseeing elections.
		\item If the President wishes to re-run for any position, then the Vice-President Internal shall act as Election Chair. 
		\item In the event that both the President and Vice-President Internal wish to run for any position, then the Vice-President External shall act as Election Chair. 
		\item In the event that the President, Vice-President Internal and Vice-President External wish to run for any position, the members present at the AGM shall nominate and vote on a Election Chair from the general membership.
		\item \st{Nominations for elections can be given to the Vice-President Internal before the AGM at which elections shall take place, and notice of nominees will be given at the same time notice of the AGM is given to the rest of the general membership. Alternatively, nominations will be taken from the floor during the AGM at which the elections are taking place.}\\
		\color{red}{Nominations for elections can be given to the Election Chair two weeks prior to the date of the elections}.\color{black}
		\item Voting shall be conducted by secret ballot.
		\item \color{red}{Election ballots shall always have an option to spoil the ballot.}\color{black}\\
		\st{Election ballots shall always contain the option: ``None of the above".}\\
		\st{I. A candidate must receive a plurality to win the position. The Election Chair shall not cast a vote except in the event of a tie.}\\
		\st{J. Should the option ``None of the above" receive a plurality vote, all nominees for the position in question fail to be elected.}\color{red}
		\item All members, except honorary members, are eligible to vote for all elected positions.
		\item The specific election process will be outlined in the By-Laws, in accordance with Section 6(3)(e) of the DSU Society Policy.\color{black}
		\item In the event that a nominee fails to be elected to a position, while they are no longer eligible for that specific position until the next general election, they may be eligible for another Council position, in the event of a by-election or as determined by the Council.
		\item Each member can hold at most ONE (1) executive position at a time.
	
	\end{enumerate}


\clearpage
\begin{center}
	\section*{Article 9:\\IMPEACHMENT}
	\addcontentsline{toc}{chapter}{Article 9: IMPEACHMENT}
	\vspace{12px}
\end{center}
\label{impeachment}
	\renewcommand{\theenumi}{\Alph{enumi}}
	\begin{enumerate}
	
		\item No member of the executive may be recalled without just cause, or in any manner not specified in this constitution.
		\item Just cause is defined by this constitution as:
			\begin{enumerate}[i.]
				\item Failing to attend FOUR (4) consecutive executive or general meetings.
				\item Failing to fulfil the majority of their mandates as dictated by this constitution, within reasonable limits, and without reasonable excuse.
				\item Conduct likely to result de-ratification of the Society, or conduct likely to bring the Society into disrepute.
				\item In the case of the Vice-President External, failing to submit a request for Ratification before November 1st of the year of their Vice-Presidency without reasonable excuse.
				\item Other gross misconduct.
			\end{enumerate}

		\item Impeachment shall be a two step process:
			\begin{enumerate}[i.]
				\item Introduction of the impeachment vote:
					\begin{enumerate}[(a)]
						\item Any member of the Council may put forth a motion for an impeachment vote; or,
						\item Any member of the Society may put forth a motion to impeach provided they have a petition signed by 25\% of current members, and this petition can also serve to request an AGM as directed by Article 7(C)(iii).
					\end{enumerate}
					
				\item The Impeachment vote:
					\begin{enumerate}[(a)]
						\item A vote to impeach a Council member can only be decided at an AGM (as directed by Article 7(C)(iii)).
						\item Advance notice of a motion to impeach must be given to the member up for impeachment, and that member must have the ability to speak in their defence before the vote is taken.
						\item The notice of the AGM shall include any motions for impeachment.
						\item A vote of $ (2/3)^{\mathrm{rd}} $ of the membership present and voting is required to impeach a Council member. Voting will be conducted by secret ballot. 
					\end{enumerate}
			\end{enumerate}
	\end{enumerate}


\clearpage
\begin{center}
	\section*{Article 10:\\RESIGNATIONS AND VACANCIES}
	\addcontentsline{toc}{chapter}{Article 10: RESIGNATIONS AND VACANCIES}
	\vspace{12px}
\end{center}
\label{resignVacancies}
	\renewcommand{\theenumi}{\Alph{enumi}}
	\begin{enumerate}
	
		\item All resignations of Council members shall be made in writing, addressed to the president, and submitted at the next regular meeting of Council for action therein. In the case of the President's resignation, the letter shall be addressed to the Vice-President Internal.
		\item Should an elected position not be filled during the initial elections, the position is to be filled by the newly elected Council. A secret ballot vote is to be held at the next Council meeting. This elected member has the same rights and responsibilities as if they had been elected in the initial elections.
		\item Should a vacancy occur in the Council the runner up in the elections will be appointed, provided they received more votes than the “None of the above” option. If no one is available for the position, the Council shall as soon as reasonably possible appoint an elections officer for the sole purpose of overseeing a by-election. Council shall open a nomination period of not less than TWO (2) days, followed by an optional day of campaigning, and, finally, ONE (1) day of voting. Voting shall in all other respects conform as nearly as is reasonably possible to Article 8.
	
	\end{enumerate}


\clearpage
\begin{center}
	\section*{Article 11:\\CONSTITUTIONAL AMENDMENTS}
	\addcontentsline{toc}{chapter}{Article 11: CONSTITUTIONAL AMENDMENTS}
	\vspace{12px}
\end{center}
\label{amendments}

	\renewcommand{\theenumi}{\Alph{enumi}}
	\begin{enumerate}
	
		\item In the event that the Society cannot be ratified as a result of the content of this constitution, the current Executive have the ability to make such changes as are deemed necessary by the Vice-President Internal of the Dalhousie Student Union. These changes become official upon ratification.
			\begin{enumerate}[i.]
				\item If any member of the Society wishes to contest the changes, they may do so at the next AGM, or petition the President to call one as per Article 7(C)(iii).
			\end{enumerate}
		
		\item With the exception of circumstances in Article 11(A), amendments to this constitution are subject to the following regulations:
			\begin{enumerate}[i.]
				\item Amendments to the constitution can only be made at \st{an AGM} \color{red} a general meeting, which could be an AGM or a special general meeting.\color{black}
				\item Motions to amend the Constitution must be presented \color{red}{by interested members of the general membership} \color{black} to the Vice-President \color{red}{Internal} \color{black} of the Society no less than one-week before the \st{AGM} \color{red} general meeting \color{black} at which the constitution will be amended.
				\item \st{All members are entitled to vote on amendments to the constitution.}\\
				\color{red}{All voting members of the general membership, except honorary members, are entitled to vote on amendments to the constitution.}\color{black}
				\item Voting will be conducted via a show of hands, though any member may request that the motion be voted on by secret ballot. If any one member requests this for any motion, voting must be done by secret ballot.
				\item A $ (2/3)^{rd} $ vote \color{red}{of those present and voting, including proxies,} \color{black} is required to pass a motion to amend the constitution.
			\end{enumerate}
	
	\end{enumerate}


\clearpage
\begin{center}
	\section*{Article 12:\\FINANCES}
	\addcontentsline{toc}{chapter}{Article 12: FINANCES}
	\vspace{12px}
\end{center}
\label{finances}

	\renewcommand{\theenumi}{\Alph{enumi}}
	\begin{enumerate}
	
		\item As stated in Article 4, the President, Vice-President Internal, and Treasurer will be the signing officers of the Society.
		
		\item At least two signing officers must sign every cheque from the bank account of the Society.
		
		\item An executive officer cannot sign a cheque made out to his or her self.
		
		\item Each newly elected Executive is responsible for setting a budget within the first month after the Council takes office.
		
		\item The budget shall be created each year by the Treasurer, and shall be passed by the executive at an Executive meeting to the Council for approval. The Council shall then motion to approve the budget at a general Council meeting.
		
		\item No less than TWENTY-FIVE (25) percent of the estimated total annual budget shall be set aside as an Emergency Fund. This money may not be accessed without a Council vote and the reasons for accessing these funds shall be reviewed by an appointed Committee as described in Article 15.
		
		\item The estimated total annual budget shall be based on the most accurate registration information available for that year as well as all remaining funds from the previous year, including the Emergency Fund.
		
		\item There should be no carry over of any deficit from one semester to the next. In the case that there is a deficit, the deficit incurred must be explained and at the least, steps shall be taken to balance the next semester's budget.
		
		\item In situations where a deficit exists when a new Executive is entering, the previous Council shall make available all Council funds, including unspent Emergency Funds, to help reduce or eliminate said deficit. The previous Council Executive shall be required to defend their reasons for incurring any such deficit to the general membership of the Society at the AGM. In the situation where a surplus exists at the end of a term, all remaining funds shall be passed on to the incoming Council.
		
		\item Refer to the Dalhousie Student Union's Treasurer's Handbook for proper procedures.
	
	\end{enumerate}


\clearpage
\begin{center}
	\section*{Article 13:\\COUNCIL TRANSITIONS}
	\addcontentsline{toc}{chapter}{Article 13: COUNCIL TRANSITIONS}
	\vspace{12px}
\end{center}
\label{}

	\renewcommand{\theenumi}{\Alph{enumi}}
	\begin{enumerate}
	
		\item End of the Year Reports are to be submitted to the transition officer via electronic mail or other reasonable method of their choosing prior to the end of their term in office. The transition officer may set an earlier deadline of their choosing, no earlier than the last meeting of the Council.
		
		\item Non-voting Council members are exempt from composing an End of the Year Report.
		
		\item Voting Council members shall not receive any honoraria for fulfilling their duties on the Council.
		
		\item Status as a Councillor shall not affect qualification for any Society reimbursement or financial support for attending conferences or workshops.
		
	\end{enumerate}


\clearpage
\begin{center}
	\section*{Article 14:\\ELECTRONIC VOTING ON MOTIONS}
	\addcontentsline{toc}{chapter}{Article 14: ELECTRONIC VOTING ON MOTIONS}
	\vspace{12px}
\end{center}
\label{}

	\renewcommand{\theenumi}{\Alph{enumi}}
	\begin{enumerate}
	
		\item Voting on motions can be done via online ballot using a method that authenticates voting members of Council.

		\item Voting must adhere to Article 7(C)(i).
		
		\item The text of the motion to be voted via online ballot shall be posted SEVENTY-TWO (72) hours before the vote.
		
		\item The voting period shall be open for at least TWENTY-FOUR (24) hours.
		
		\item If the vote does not pass, the motion will be presented at the next general Council meeting.
	
	\end{enumerate}


\clearpage
\begin{center}
	\section*{Article 15:\\COMMITTEES}
	\addcontentsline{toc}{chapter}{Article 15: COMMITTEES}
	\vspace{12px}
\end{center}
\label{}

	When a Committee is formed as mandated by this Constitution, the following conditions shall be met:\\
	
	\renewcommand{\theenumi}{\Alph{enumi}}
	\begin{enumerate}
	
		\item A minimum of THREE (3) members; and,
		
		\item A majority of the Committee must not be on Council.
		
	\end{enumerate}


\clearpage
\color{red}{
\begin{center}
	\section*{Article 16:\\DESCRIPTION OF BY-LAWS}
	\addcontentsline{toc}{chapter}{Article 16: DESCRIPTION OF BY-LAWS}
	\vspace{12px}
\end{center}
\label{}

	By-Laws shall have the following characteristics:\\
	
	\renewcommand{\theenumi}{\Alph{enumi}}
	\begin{enumerate}
	
		\item They shall regulate events and issues in order to facilitate the routine operation of the Society; and
		\item Where applicable, all effects and consequences shall be subject first to the Articles of this constitution.
		
		
	\end{enumerate}
}

\clearpage
\color{red}{
\begin{center}
	\section*{BY-LAWS}
	\addcontentsline{toc}{chapter}{BY-LAWS}
	\vspace{12px}
\end{center}
\label{}

	\renewcommand{\theenumi}{\Alph{enumi}}
	\begin{enumerate}
	
		\item Amendments to By-Laws
			\begin{enumerate}[i.]
				\item The general membership must be notified of any motion to change the By-Laws, including the addition of new By-Laws or amendments to existing By-Laws, at least ONE WEEK prior to the date when the motion will be called to a vote.
				\item Addition of new By-Laws or amendments to the existing By-Laws may be passed:
					\begin{enumerate}[(a)]
						\item With a $ (2/3)^{rd} $ majority of those present and voting, including proxies, in general council meetings; or,
						\item With a simple ($ 50\% + 1 $) majority of those present and voting, including proxies, at a general meeting (including special general meetings or an AGM).
					\end{enumerate}
			\end{enumerate}
	
		\item Elections for Executive and Council Positions
			\begin{enumerate}[i.]
				\item Members may be nominated for any of the following positions, subject to Article 4(E), Article 4(F) and Article 8 of this constitution:
				\begin{enumerate}[a.]
					\item The Executive
					\item PhD Representative
					\item MCS Representative
					\item MACS Representative
					\item MHI Representative
					\item MEC Representative
					\item Member-At-Large
				\end{enumerate}
				\item The executive voting ballot shall allow members to select between zero (0) and five (5) candidates for positions on the executive. 
				\item The five (5) candidates with the most votes shall form the executive.
				\item If fewer than five (5) members are nominated for the executive positions, or if fewer than five (5) members are elected from the set of nominees, then they shall form the executive.
				\item The newly elected council shall vote at its first meeting, which members of the executive shall fill specific positions within the executive (see Article 5).
				\item The non-executive council positions shall be decided by a simple majority.
				\item If there are vacant positions on the council, whether due to circumstances described in By-Law B(iv) or due to lack of nominees, the newly elected council shall announce by-elections for filling the positions after they assume office.
				\item Members shall have the option to choose ``None of the above'' for non-executive council positions
			\end{enumerate}
	
		\item Dalhousie Computer Science In-house (DCSI) Conference
			\begin{enumerate}[i.]
				\item Each year in advance of the Conference, two co-chairs shall be selected to form the Organizing Committee:
				\begin{enumerate}[a.]
					\item The first co-chair shall be selected in a general council meeting from among the active members of council; and,
					\item The second co-chair shall be selected by the first co-chair from among the general membership who do not hold a council position; and,
					\item Where applicable, preference in the selection of both co-chairs should be given to those individuals who have served on previous DCSI Organizing Committees.
				\end{enumerate}
				\item To form the Organizing Committee, the two co-chairs will present an open call for volunteers to the general membership.
				\item All funds and assets belonging to the DCSI conference shall be managed by the DCSI Organizing Committee, with excess amounts held in reserve for subsequent years.
			\end{enumerate}
	
	\end{enumerate}
}

%\clearpage
%\begin{center}
%	\section*{Article 1X:\\NAME}
%	\vspace{12px}
%\end{center}
%\label{}
%
%	\renewcommand{\theenumi}{\Alph{enumi}}
%	\begin{enumerate}
%	
%		\item 
%	
%	\end{enumerate}

\clearpage
\centering
\begin{titlepage}

	\begin{table}[t]
	\renewcommand{\arraystretch}{1.3}
	\centering
	\begin{tabular}{c}
		Dalhousie Computer Science Graduate Society (CSGS)\\2015
	\end{tabular}
	\end{table}

\end{titlepage}


\end{document}